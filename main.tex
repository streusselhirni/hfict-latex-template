\documentclass{hfscript}
\usepackage{hyperref}       % Für klickbare Links in PDF
\usepackage{booktabs}       % Für schönere Tabellen mit top-/mid-/bottomrule
\usepackage{tabularx}       % Für Tabellen mit fixer Breite
\usepackage{multirow}       % Für Tabellenzellen über mehrere Spalten/Zeilen
\usepackage{textcomp}       % Für Symbole, zb. Copyright
\usepackage{lastpage}       % Für \lastpage
\usepackage{graphicx}       % Für Bilder
\usepackage{amsmath}        % Für Mathematik
\usepackage{amssymb}        % Für Mathesymbole
\usepackage{pifont}         % Einige Mathe-Befehle benötigen dieses Paket für Symbole
\usepackage{cancel}         % Zum Durchstreichen mit \cancel,\bcancel,\xcancel
\usepackage{floatrow}       % Bilder und Tabellen nebeneinander
\usepackage{placeins}       % Für \FloatBarrier
\usepackage{wrapfig}        % Für Bilder mit Text drumrum
\usepackage{pdfpages}       % PDF Seiten einfügen

%% Wenn ein Quellenverzeichnis genutzt wird, die folgende Zeile unkommentieren
\usepackage{csquotes}               % Biblatex verlangt dieses Paket.
\usepackage[style=apa]{biblatex}    % Für Quellenverzeichnis
\DeclareLanguageMapping{ngerman}{ngerman-apa}

%% Zwingende Variablen, damit Vorlage korrekt funktioniert
\title{Titel der Arbeit}
\subtitle{Ein Untertitel}
\hfsem{Herbstsemester 2017}
\hffach{TE-Templating}
\dozent{Streusselhirni}
\author{Nicolas Haenni}
\date{Dezember 2017}

\begin{document}
%% Titelseite / Erste Seite
%% Page Style anders, damit Kopf-/Fußzeilen passen
\thispagestyle{scrheadings}
%%%%%%%%%%%%%%%%%%%%%%%%%%%%%%%%%%%%%%%%%%%%%%%%%%%%%%%%%%
%% Titel-Stil

%% Short Title
%% Titel ist nur oben auf der ersten Seite, Inhalt kann
%% direkt auf der ersten Seite starten.

\input{parts/short_title}

%% Long Title
%% Es gibt eine Titelseite. Inclusive Abstrakt 
%% (Zusammenfassung), Dokumenten-Management und
%% Lizenzangabe.

%\maketitle
\begin{abstract}
    Eine Zusammenfassung über dieses Dokument.
\end{abstract}

\subsection*{Dokumentenhistorie}
\begin{center}
    \begin{tabularx}{.9\textwidth}{llX}
        \toprule
        \textbf{Version} & \textbf{Datum} & \textbf{Beschreibung} \\
        \midrule
        0.1 & 01.02.2018 & Erstellen des Dokumentes \\
        \bottomrule
    \end{tabularx}
\end{center}

\vfill

\footnotesize{Dieses Dokument steht unter der GNU General Public Free Document License (GPL FDL). Weitere Informationen unter https://www.gnu.org/licenses/fdl.html.}
\clearpage
%%%%%%%%%%%%%%%%%%%%%%%%%%%%%%%%%%%%%%%%%%%%%%%%%%%%%%%%%%
%% Inhaltsverzeichnis
%% Wenn ein Inhaltsverzeichnis benötigt wird, die folgende
%% Zeile unkommentieren
%% Für Seitenumbruch nach dem Inhaltsverzeichnis beide
%% Zeilen unkommentieren.
%\tableofcontents
%\clearpage

% Separate Kapitel können per input eingebunden werden.
\section{Beispielkapitel}
Dies ist ein Beispielkapitel. Viel Spaß damit!

% Wenn die Kapitel per include eingebunden werden, starten sie immer auf einer neuen Seite
\section{Dies ist noch ein Beispiel}
Hier ist ein zweites Beispielkapitel.

%%%%%%%%%%%%%%%%%%%%%%%%%%%%%%%%%%%%%%%%%%%%%%%%%%%%%%%%%%
%% Anhänge
%% Falls Figures/Tables benutzt werden, können hier die Verzeichnisse aktiviert werden.
%\listoffigures
%\listoftables

%% Wenn ein Quellenverzeichnis genutzt wird, die folgende Zeile ebenfalls unkommentieren
%\printbibliography[title=Quellenverzeichnis]
\end{document}
